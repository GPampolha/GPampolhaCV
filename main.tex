%%%%%%%%%%%%%%%%%
% This is an sample CV template created using altacv.cls
% (v1.3, 10 May 2020) written by LianTze Lim (liantze@gmail.com). Now compiles with pdfLaTeX, XeLaTeX and LuaLaTeX.
% (v1.7.1b, 11 Jan 2024) forked by Nicolás Omar González Passerino (nicolas.passerino@gmail.com)
%
%% It may be distributed and/or modified under the
%% conditions of the LaTeX Project Public License, either version 1.3
%% of this license or (at your option) any later version.
%% The latest version of this license is in
%%    http://www.latex-project.org/lppl.txt
%% and version 1.3 or later is part of all distributions of LaTeX
%% version 2003/12/01 or later.
%%%%%%%%%%%%%%%%

%% If you need to pass whatever options to xcolor
\PassOptionsToPackage{dvipsnames}{xcolor}

%% If you are using \orcid or academicons
%% icons, make sure you have the academicons
%% option here, and compile with XeLaTeX
%% or LuaLaTeX.
% \documentclass[10pt,a4paper,academicons]{altacv}

%% Use the "normalphoto" option if you want a normal photo instead of cropped to a circle
% \documentclass[10pt,a4paper,normalphoto]{altacv}

%% Fork (before v1.6.5a): CV dark mode toggle enabler to use a inverted color palette.
%% Use the "darkmode" option if you want a color palette used to 
\documentclass[10pt,a4paper,ragged2e,withhyper,darkmode, normalphoto]{altacv}

% \documentclass[10pt,a4paper,ragged2e,withhyper]{altacv}

%% AltaCV uses the fontawesome5 and academicons fonts
%% and packages.
%% See http://texdoc.net/pkg/fontawesome5 and http://texdoc.net/pkg/academicons for full list of symbols. You MUST compile with XeLaTeX or LuaLaTeX if you want to use academicons.

%% Fork v1.6.5c: Overwriting sloppy environment to ignore any spaces and be used to solve overlapping cvtags
\newenvironment{sloppypar*}{\sloppy\ignorespaces}{\par}

% Change the page layout if you need to
\geometry{left=1.2cm,right=1.2cm,top=1cm,bottom=1cm,columnsep=0.75cm}

% The paracol package lets you typeset columns of text in parallel
\usepackage{paracol}

% Change the font if you want to, depending on whether
% you're using pdflatex or xelatex/lualatex
\ifxetexorluatex
  % If using xelatex or lualatex:
  \setmainfont{Roboto Slab}
  \setsansfont{Lato}
  \renewcommand{\familydefault}{\sfdefault}
\else
  % If using pdflatex:
  \usepackage[rm]{roboto}
  \usepackage[defaultsans]{lato}
  % \usepackage{sourcesanspro}
  \renewcommand{\familydefault}{\sfdefault}
\fi

% Fork (before v1.6.5a): Change the color codes to test your personal variant on any mode
\ifdarkmode%
  \definecolor{PrimaryColor}{HTML}{FFFFFF}
  \definecolor{SecondaryColor}{HTML}{FFFFFF}
  \definecolor{ThirdColor}{HTML}{000C84}
  \definecolor{BodyColor}{HTML}{ABABAB}
  \definecolor{EmphasisColor}{HTML}{ABABAB}
  \definecolor{BackgroundColor}{HTML}{191919}
\else%
  \definecolor{PrimaryColor}{HTML}{001F5A}
  \definecolor{SecondaryColor}{HTML}{0039AC}
  \definecolor{ThirdColor}{HTML}{F3890B}
  \definecolor{BodyColor}{HTML}{666666}
  \definecolor{EmphasisColor}{HTML}{2E2E2E}
  \definecolor{BackgroundColor}{HTML}{E2E2E2}
\fi%

\colorlet{name}{PrimaryColor}
\colorlet{tagline}{SecondaryColor}
\colorlet{heading}{PrimaryColor}
\colorlet{headingrule}{ThirdColor}
\colorlet{subheading}{SecondaryColor}
\colorlet{accent}{SecondaryColor}
\colorlet{emphasis}{EmphasisColor}
\colorlet{body}{BodyColor}
\pagecolor{BackgroundColor}

% Change some fonts, if necessary
\renewcommand{\namefont}{\LARGE\rmfamily\bfseries}
\renewcommand{\personalinfofont}{\small\bfseries}
\renewcommand{\cvsectionfont}{\LARGE\rmfamily\bfseries}
\renewcommand{\cvsubsectionfont}{\large\bfseries}

% Change the bullets for itemize and rating marker
% for \cvskill if you want to
\renewcommand{\itemmarker}{{\small\textbullet}}
\renewcommand{\ratingmarker}{\faCircle}

%% sample.bib contains your publications
%% \addbibresource{main.bib}

\begin{document}
    \name{Gabriel Pampolha}
    \tagline{Estudante universitário}
    %% You can add multiple photos on the left or right
    \photoL{5cm}{gpampolha}

    \personalinfo{
        \email{pampolhagabriel.gp@gmail.com}\smallskip
        \phone{21994566835}
        \location{Rio de Janeiro, Brasil}\\
        %\homepage{nicolasomar.me}
        %\medium{nicolasomar}
        %% You MUST add the academicons option to \documentclass, then compile with LuaLaTeX or XeLaTeX, if you want to use \orcid or other academicons commands.
        % \orcid{0000-0000-0000-0000}
        %% You can add your own arbtrary detail with
        %% \printinfo{symbol}{detail}[optional hyperlink prefix]
        % \printinfo{\faPaw}{Hey ho!}[https://example.com/]
        %% Or you can declare your own field with
        %% \NewInfoFiled{fieldname}{symbol}[optional hyperlink prefix] and use it:
        % \NewInfoField{gitlab}{\faGitlab}[https://gitlab.com/]
        % \gitlab{your_id}
    }
    
    \makecvheader
    %% Depending on your tastes, you may want to make fonts of itemize environments slightly smaller
    % \AtBeginEnvironment{itemize}{\small}
    
    %% Set the left/right column width ratio to 6:4.
    \columnratio{0.28}

    % Start a 2-column paracol. Both the left and right columns will automatically
    % break across pages if things get too long.
    
    \begin{paracol}{2}      
    
        % ----- HABILIDADES -----
        \cvsection{HABILIDADES}
            \begin{sloppypar*}
                \cvtags{Pensamento crítico/true, Trabalho em equipe/true, Solução de problemas/true, Motivação/true, Empatia/true, Liderança/true, Adaptabilidade/true}
            \end{sloppypar*}
        % ----- HABILIDADES ----

        % ----- CONHECIMENTO -----
        \cvsection{CONHECIMENTO}
            \begin{sloppypar*}
                \cvtags{Python/true, Office 365/true, Banco de dados/true,  Programação em C/true, Desenvolvimento web/true, Computação em nuvem/true}
            \end{sloppypar*}
        % ----- CONHECIMENTO -----

        % ----- INTERESSES -----
        \cvsection{INTERESSES}
            \begin{sloppypar*}
                \cvtags{Matemática/true, Física/true,Astrofísica/true, Galáxias/true, Computação/true, Data science/true}
            \end{sloppypar*}
        % ----- INTERESSES -----
        
        % ----- IDIOMAS -----
        \cvsection{IDIOMAS}
            \cvlang{Português}{Nativo}\\
            \divider

            \cvlang{Inglês}{Fluente}\\
            \divider

            \cvlang{Espanhol}{Básico}
        % ----- IDIOMAS -----
        
        % use ONLY \newpage if you want to force a page break for
        % ONLY the current column
        \newpage
        
        %% Switch to the right column. This will now automatically move to the second
        %% page if the content is too long.
        \switchcolumn
        
        % ----- RESUMO -----
        \cvsection{RESUMO}
            \begin{quote}
                Bolsista de iniciação científica PIBIC/CNPq, desenvolvendo projeto de pesquisa na área de Astrofísica Extragaláctica, com foco em histórico de formação de galáxias próximas com o instrumento SPARC4 do telescópio Perkin-Elmer, situado no Observatório do Pico dos Dias,  Minas Gerais.
            \end{quote}
        % ----- RESUMO -----
        
        % ----- EDUCAÇÃO -----
        \cvsection{EDUCAÇÃO}
            \cvevent{Ensino Médio}{Colégio Militar do Rio de Janeiro}{Fev/2015 -- Nov/2017}{Rio de Janeiro, Brasil}
            \divider
            
            \cvevent{Bacharelado em Sistemas de Informação}{UNESA}{Fev/2021 -- Atualmente}{Rio de Janeiro, Brasil}
            \divider

            \cvevent{Graduação em Astronomia}{UFRJ}{Mar/2022 -- Atualmente}{Rio de Janeiro, Brasil}
        % ----- EDUCAÇÃO -----

        % ----- EXPERIÊNCIA -----
        \cvsection{EXPERIÊNCIA}
            \cvevent{Membro}{Minerva Rockets e Sats} {Jan/2023 -- Nov/2023}{Rio de Janeiro, Brasil}
            \divider
            
            \cvevent{Monitor}{Física Experimental I}{Mar/2023 -- Dez/2023}{Rio de Janeiro, Brasil}
        % ----- EXPERIÊNCIA -----
        
        % ----- CURSOS LIVRES -----
        \cvsection{CURSOS LIVRES}  
        
            \cvevent{Sistema Solar}{Observatório do Valongo}{Nov/22}{Rio de Janeiro, Brasil}
            \divider

            \cvevent{Python 3}{Curso em Vídeo}{Nov/22}{Online}
        % ----- CURSOS LIVRES -----
        
    \end{paracol}
\end{document}